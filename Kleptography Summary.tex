%% LyX 2.2.3 created this file.  For more info, see http://www.lyx.org/.
%% Do not edit unless you really know what you are doing.
\documentclass[english]{article}
\usepackage{mathpazo}
\usepackage[scaled=0.95]{helvet}
\usepackage{courier}
\usepackage[T1]{fontenc}
\usepackage[latin9]{inputenc}
\usepackage{babel}
\begin{document}

\title{Kleptography: Using Cryptography against Cryptography}

\author{A Summary by Kevin Burk and Ashwini Patil}

\maketitle
This paper introduces Kleptography and how to use cryptographic techniques
against the Diffie-Hellman protocol. Kleptography, as defined in the
paper is the \textquotedbl{}study of stealing information securely
and subliminally.\textquotedbl{} The authors had previously introduced
the concept of SETUP (Secretly Embedded Trapdoor with Universal Protection),
which was designed such that it gives a unique advantage to the attacker.
A SETUP is an algorithm that can be used to leak encrypted secret
key information, when present within a cryptosystem. They discuss
this concept again, in the context of Diffie-Hellman key exchanges.
The paper can be applied to applications such as chip-and-pin card
transactions and generation of RSA keys. We will be talking about
how \textquotedbl{}black box\textquotedbl{} devices like USB tokens
or chip cards can be designed to leak their keys in such a way that
only the designer can recover them - or indeed know that keys are
being leaked at all.

The authors discuss the Diffie-Hellman key exchange protocol, and
further go on to prove how it can be compromised using a strong attack
based on the discrete logarithm problem. They also discuss the security
of this attack from the attacker's point of view. The attacker can
steal secrets in an exclusive and subliminal manner, implying that
the attack is not traceable by anyone but the attacker. The concept
of a strong SETUP is introduced in this paper and further used to
attack the Diffie-Hellman key exchange protocol, in comparison to
earlier used weak SETUPs. Here lies the novelty of this paper. Earlier
attacks employing weak SETUPs were cryptographically insecure in the
sense that the owner of the device was able to detect that his/her
secret was compromised. The idea of m out of n (bits) leakage bandwidth
is also a new concept in this paper. The authors then go on to discuss
the Probabilistic Bias Removal Method. It presents a method to obtain
larger uniform random numbers for RSA key generation.

While going through the paper, we were very intrigued by the wide
range of applications that the paper's concepts could be applied to.
Additionally, the fact that anything with extra code present could
be accessing your secret information, without your knowledge, was
astounding. It is believed that the Dual\_EC\_DRBG cryptographically
secure random number generator, is with only minor modifications,
equivalent to the Young-Yung backdoor from this paper, presented at
Eurocrypt in 1997. While the paper was very interesting, one thing
that we disliked about the paper was the way certain variables were
randomly used in mathematical proofs without them being defined earlier.
However, we were able to figure out what the variables represented
from their significance in the proofs, and this did not pose to be
much of an issue with understanding what the paper is trying to convey.

In conclusion, the paper presented a well-rounded proof of the attack
(with subsequent theorems included in the paper) and how it works
on the Diffie-Hellman key exchange protocol. We definitely recommend
reading this paper. 
\end{document}
